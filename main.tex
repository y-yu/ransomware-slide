\definecolor{links}{HTML}{2A1B81}
\hypersetup{colorlinks,linkcolor=,urlcolor=links}

\usetheme{Boadilla}
\usecolortheme{seahorse}
\usefonttheme{serif}
\beamertemplatenavigationsymbolsempty

\setbeamertemplate{bibliography item}{\insertbiblabel}
\setbeamersize{description width=1cm}

\usepackage{luacode}
\usepackage{luatexja}
\usepackage{pgfpages}

\begin{luacode*}
  USE_IPAFONT = os.getenv"USE_IPAFONT"
  USE_YUFONT = os.getenv"USE_YUFONT"
  
  if USE_YUFONT == "true" then
    tex.sprint("\\AtBeginDocument{\\usepackage[yu-osx, deluxe, expert]{luatexja-preset}}")
  elseif USE_IPAFONT == "true" then
    tex.sprint("\\AtBeginDocument{\\usepackage[ipaex, deluxe, expert]{luatexja-preset}}")
  else
    tex.sprint("\\AtBeginDocument{\\usepackage[hiragino-pro, deluxe, expert]{luatexja-preset}}")
  end
\end{luacode*}

\usepackage{epigraph}
\usepackage{etoolbox}
\usepackage{tikz}
\usepackage{framed}
\usepackage{libertine}
\usepackage{amsmath}
\usepackage{mathtools}
\usepackage{listings}
\usepackage{tikz-qtree}

\renewcommand{\kanjifamilydefault}{\gtdefault}

%\setbeameroption{show notes on second screen=right}

\setmainfont[Ligatures=TeX]{Linux Libertine O}
\setsansfont[Ligatures=TeX]{CMU Sans Serif}
\setmonofont[Ligatures=TeX]{CMU Typewriter Text}

\input{vc.tex}

\title[公平なランサムウェア]{%
  公平なランサムウェア \\
  {\large The Fairest Ransomware}
}
\author[吉村 優]{%
  吉村 優 \\
  Hikaru \textsc{Yoshimura}
}
\date[June 24, 2017]{%
  June 24, 2017 \\%
  {\footnotesize (Commit ID: \GITAbrHash)}
}
\institute[株式会社ドワンゴ]{%
  株式会社ドワンゴ \\
   \href{mailto:yyu@mental.poker}{yyu@mental.poker}
}

\input{./lib/quotebox.tex}
\input{./lib/footnotemark.tex}
\input{./lib/ballon.tex}
\newfontfamily\listingfont{Menlo}
\definecolor{dkgreen}{rgb}{0,0.6,0}
\definecolor{gray}{rgb}{0.5,0.5,0.5}
\definecolor{mauve}{rgb}{0.58,0,0.82}

\makeatletter
\lst@CCPutMacro\lst@ProcessOther {"2D}{\lst@ttfamily{-{}}{-{}}}
\@empty\z@\@empty
\makeatother

\lstset{
  basicstyle=\listingfont,
  frame=single,
  xleftmargin=2em,
  xrightmargin=1em,
  breaklines=true
}

\lstdefinestyle{scala}{
  basicstyle=\listingfont\scriptsize,
  breakatwhitespace=false,
  language=scala,
  captionpos=b,
  commentstyle=\listingfont\scriptsize\color{dkgreen},
  extendedchars=true,
  xleftmargin=2em,
  xrightmargin=1em,
  keepspaces=true,
  keywordstyle=\listingfont\scriptsize\color{blue},
  emphstyle=\listingfont\scriptsize\color{cyan},
  rulecolor=\listingfont\scriptsize\color{black},
  showspaces=false,
  showstringspaces=false,
  showtabs=false,
  stringstyle=\listingfont\scriptsize\color{mauve},
  tabsize=2
}

\lstdefinelanguage{scala}{
  morekeywords={abstract,case,catch,class,def,%
    do,else,extends,false,final,finally,%
    for,if,implicit,import,match,mixin,%
    new,null,object,override,package,%
    private,protected,requires,return,sealed,%
    super,this,throw,trait,true,try,%
    type,val,var,while,with,yield},
  moreemph={Byte,Short,Int,Long,Float,Double,Char,
    String,Boolean,Unit,Null,Nothing,Any,AnyRef,
    Left,Right,Either},
  otherkeywords={=>,<-,<\%,<:,>:,\#,@},
  sensitive=true,
  morecomment=[l]{//},
  morecomment=[n]{/*}{*/},
  morestring=[b]",
  morestring=[b]',
  morestring=[b]"""
}


\newcommand\ballref[1]{%
\tikz \node[circle, shade,ball color=structure.fg,inner sep=0pt,%
  text width=8pt,font=\tiny,align=center] {\color{white}\ref{#1}};
}

\begin{document}

\frame{\maketitle}

\begin{frame}
  \frametitle{目次}

  \tableofcontents
\end{frame}

\section{自己紹介}
\begin{frame}
  \frametitle{自己紹介}
  
  \begin{columns}
    \begin{column}{0.3\textwidth}
      \centering
      \begin{figure}
        \includegraphics[width=0.95\textwidth]{img/bird2x.png}
      \end{figure}

      \begin{description}
        \item[Twitter] \href{https://twitter.com/\_yyu\_}{@\_yyu\_}
        \item[Qiita] \href{http://qiita.com/yyu}{yyu}
        \item[GitHub] \href{https://github.com/y-yu}{y-yu}
      \end{description}
    \end{column}
    \begin{column}{0.7\textwidth}
      \begin{itemize}
        \item<2-> 筑波大学 情報科学類卒(学士)
        \item<3-> 株式会社ドワンゴ 入社
        \item<4-> DCS部 認証基盤セクション
        \item<5-> 暗号技術に興味があります
      \end{itemize}
    \end{column}
  \end{columns}
\end{frame}

\section{はじめに}

\begin{frame}
  \frametitle{はじめに}

  \begin{itemize}
    \item<2-> 最近は\textbf{ランサムウェア}が流行している
  \end{itemize}

  \uncover<3->{
    \begin{block}{ランサムウェア}
      被害者のコンピューター内のデータを暗号化して、
      復号の対価としてBitcoinなどを支払わせるマルウェアのこと。
    \end{block}
  }

  \uncover<4->{
    \begin{exampleblock}{代表的なランサムウェア}
      \begin{itemize}
        \item WannaCry
        \item Petya
        \item GoldenEye 
      \end{itemize}
    \end{exampleblock}
  }

  \begin{center}
    \uncover<5->{
      \begin{tikzpicture}
        \calloutquote[width=9cm,position={(0.7,-0.2)},fill=red!20,rounded corners]{Bitcoinを支払ったら本当に復号してくれるのか?}
      \end{tikzpicture}
    }
  \end{center} 
\end{frame}

\section{定義と記法}

\begin{frame}
  \frametitle{公平なランサムウェア}

  \uncover<2->{
    \begin{shadequote}{}
      Bitcoinを支払えば、元のデータが被害者の納得した確率で
      復号されるランサムウェアのこと。
    \end{shadequote}
  }

    \begin{center}
    \uncover<3->{
      \begin{tikzpicture}
        \calloutquote[width=7cm,position={(0.7,-0.2)},fill=blue!20,rounded corners]{暗号技術を使えばできる}
      \end{tikzpicture}
    }
  \end{center} 
\end{frame}

\newcommand{\Enc}[2]{\text{Enc}_{#1}\left(#2\right)}
\newcommand{\Dec}[2]{\text{Dec}_{#1}\left(#2\right)}

\begin{frame}
  \frametitle{定義}

  \begin{description}
    \item<2->[対称鍵暗号] AESのような、
    共通の\textbf{対称鍵}を利用して暗号化・復号を行う暗号方式のこと。
    暗号化関数を$\text{Enc}$、復号関数を$\text{Dec}$とすると、次がなりたつ。
    \[
      x = \Dec{k}{\Enc{k}{x}}
    \]
    
    \item<3->[RSA暗号] 暗号化に用いる\textbf{公開鍵}と
    復号に用いる\textbf{秘密鍵}が異なる暗号方式のこと。
    公開鍵$(e, N)$と秘密鍵$d$とすると、次がなりたつ。
    \[
      x = (x^e)^d = (x^d)^e \pmod{N}
    \]

    \item<4->[ゼロ知識証明] 証明者は証明したい命題以外の情報を
    秘密にしたまま命題を証明できる
    
  \end{description}
\end{frame}

\begin{frame}
  \centering
  {\Huge Thank you for your attention!\\
    \vspace{1em}
    Any question?
  }
\end{frame}

\end{document}
